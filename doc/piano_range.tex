\documentclass[convert]{standalone}
\usepackage{xparse}
\usepackage{tikz}
\usepackage{musixtex}
\ExplSyntaxOn
\NewExpandableDocumentCommand \myFrequencyCalculation { m }
  { \fp_to_decimal:n { round(2**((#1-49)/12)*440,3) } }
\ExplSyntaxOff
\begin{document}
\pgfdeclarelayer{blacknotes}
\pgfsetlayers{main,blacknotes}
\tikzset{tight fit/.style={inner sep=0pt, outer sep=0pt}}

\begin{tikzpicture}
\def\lastnotenodename{clefs}
\node [text width=1cm, tight fit] (clefs) at (0,0) {
    \begin{music}
        \instrumentnumber{1}
        \instrumentnumber{2}
        \nostartrule
        \setstaffs1{1}
        \setstaffs2{1}
        \setclef1{\bass}
        \setclef2{\treble}
        \startextract
        \hskip2.0\elemskip
        \zendextract
    \end{music}
};

\foreach \note [
    evaluate={
        \n=int(mod(\note-1, 12));
	\realnote=int(\note-49);
        \octave=int((\note+8)/12);
        \t=int(floor((\note-1)/12)*7-7);
        \notename={"A","","B","C","","D","","E","F","","G",""}[\n];
        \tonicsolfa={"la","","si","do","","ré","","mi","fa","","sol",""}[\n];
        \blacknote={0,1,0,0,1,0,1,0,0,1,0,1}[\n];
        \frequency={\myFrequencyCalculation{\note}};}
] in {1,...,88}{

    \ifnum\octave>3
        \tikzset{extract anchor/.style={anchor=south west, at=(\lastnotenodename.south east)}}
    \else
        \tikzset{extract anchor/.style={anchor=north west, at=(\lastnotenodename.north east)}}
    \fi
    \ifnum\blacknote=0
        \edef\notenodename{\notename_\octave}
        \node (\notenodename) [tight fit,text width=1cm, extract anchor/.try]  {%           
            \begin{music}
                \instrumentnumber{1}
                \instrumentnumber{2}
                \nostartrule        
                \setstaffs1{1}
                \setstaffs2{1}  
                \setclefsymbol1{\empty}
                \setclefsymbol2{\empty}     
                \setclef1{\bass}
                \setclef2{\treble}                      
                \startextract
                \transpose\t
                \hskip-1.5\elemskip         
                \ifnum\octave>3
                    \ifnum\octave>4
                        \Notes \nextinstrument \ql{\notename} \en       
                    \else
                        \Notes \nextinstrument \qu{\notename} \en                       
                    \fi
                \else
                    \ifnum\octave>2
                        \Notes \ql{\notename} \en
                    \else
                        \Notes \qu{\notename} \en
                    \fi
                \fi
                \zendextract
            \end{music}
        };
        \xdef\lastnotenodename{\notenodename}       
        \node [anchor=base] (sol-fa)  at (\notenodename |- 0,-3) {\tonicsolfa$_\octave$};

        \draw (\notenodename.south west |- 0,-4) rectangle ++(1, -4);
        \node [rotate=90, font=\footnotesize, anchor=east] 
            at (\notenodename.north |- 0,-4) {\frequency};
        \node [font=\footnotesize, anchor=south]  
            at (\notenodename.south |- 0,-8) {\realnote};
        \node [font=\footnotesize, anchor=south] 
            at (\notenodename.south |- 0,-8.5)  {\notename$_\octave$};
        \draw (\notenodename.south west |- sol-fa.south) 
            rectangle (\notenodename.south east |- 0,1.125); %0.125 by trial and error
    \else
        \begin{pgfonlayer}{blacknotes}
        \fill ([xshift=-0.25cm]\lastnotenodename.north east |- 0,-4) rectangle ++(0.5, -2.5);
	\node [text=white, font=\footnotesize, anchor=south]  
            at (\lastnotenodename.south east |- 0,-6.5) {\realnote};
        \node  [rotate=90, text=white, font=\footnotesize, anchor=east]
            at (\lastnotenodename.north east |- 0,-4) {\frequency};
        \end{pgfonlayer}
    \fi
}
\node [rotate=90] at (0,-6) {Fr\`equency (Hz)};
\end{tikzpicture}

\end{document}
